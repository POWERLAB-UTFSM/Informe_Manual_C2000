Antes de comenzar a incursionar en el mundo de los microcontroladores de Texas Instruments, primero se deben descargar ciertos recursos (enlaces en negrita):
\begin{itemize}
	\item \href{https://software-dl.ti.com/ccs/esd/documents/ccs_downloads.html}{\textbf{Code Composer Studio}} Es el entorno de desarrollo (IDE) a utilizar para desarrollar aplicaciones relacionadas con los microcontroladores de TI. Está disponible para Windows, Linux, MacOS, de 64b, versiones de 32b están obsoletas.
	\item \href{http://www.ti.com/tool/download/C2000WARE}{\textbf{C2000WARE}} Es un set de software y documentación, desarrollado por TI para ayudar a programar sus microcontroladores. Incluye Librerías, drivers, ejemplos de periféricos, datasheet, pinout diagram, etc.
	\item \href{http://www.ti.com/tool/CONTROLSUITE}{\textbf{controlSUITE}}Es el software predecesor de C2000WARE, Contiene librerías, ejemplos, diagramas, datasheet, etc, con el objetivo de minimizar el tiempo de programación de los MCU de TI. Se recomienda descargar esto (a pesar de que su última actualización fue en 2018), ya que algunos ejemplos que contiene se programan de forma distinta que en C2000WARE, lo que puede enriquecer el estudio.
\end{itemize}
Luego, existen muchas formas de continuar, lo que dependerá del grado de conocimiento del usuario.La forma que recomienda TI es seguir su \href{https://training.ti.com/c2000-f2837xd-microcontroller-workshop?context=1137791-1137781}{\textbf{C2000™ F2837xD Microcontroller Workshop}}. Es un curso técnico con una serie de laboratorios en los que se explican a grandes rasgos el funcionamiento tanto de la arquitectura del MCU, el entorno de programación, inicialización de periféricos, etc.
\justify
Otra forma de iniciar es comenzar leyendo los ejemplos (ya que son plug$\&$play, solo se debe conectar el microcontrolador y funcionan) que se encuentran en C2000WARE y controlSUITE (los cuales se deben cargar en el CCS) y entender su funcionamiento, apoyando la lectura con el Technical Reference Manual (TMR de aquí en adelante). El ejemplo más sencillo es "$led\_ex1\_blinky.c$", en el que se definen la salida de uno de los pines GPIO como el led y se enciende y apaga la salida cada cierto tiempo en un loop infinito.